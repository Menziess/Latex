%%%%%%%%%%%%%%%%
%% Preambule  %%
%%%%%%%%%%%%%%%%

\documentclass[11pt]{article}

\usepackage{amsmath,amsfonts,amssymb}
\usepackage{upgreek}
\usepackage{enumerate}
\usepackage{enumitem}   %
\usepackage{multicol}
\usepackage{scrextend}
\usepackage{fancyhdr}
\usepackage{lastpage}
\usepackage{subcaption} % 2 column images
\usepackage[all]{xy}
\usepackage{tikz-qtree}
\usepackage[margin=2.5cm]{geometry}
\input{../../Template/fitch.sty}

\setlength{\parindent}{0pt}

\pagestyle{fancy}

\lhead{\opdrachtNaam\ \opdrachtNummer}
\rhead{\naam(\studentNummer)}
\rfoot{Pagina\ \thepage\ van\ \pageref{LastPage}}
\lfoot{\datum}
\cfoot{}

\renewcommand\headrulewidth{0.4pt}
\renewcommand\footrulewidth{0.4pt}

\newcommand{\E}{\exists}
\newcommand{\A}{\forall}

\newcommand{\ccen}[2]{\llap{$#1$}${}\mathrel{\circ}{}$\rlap{$#2$}}

%%%%%%%%%%%%%%
%% Gegevens %%
%%%%%%%%%%%%%%

\newcommand{\naam}          {Stefan Schenk}
\newcommand{\studentNummer} {11881798}
\newcommand{\opdrachtNaam}  {Assignment}
\newcommand{\opdrachtNummer}{5}
\newcommand{\datum}         {December 2017}

%%%%%%%%%%%%%%%%
%% Antwoorden %%
%%%%%%%%%%%%%%%%

\begin{document}

% Opgaven:
%
% 6.24, 6.26, 6.30
% extra 2b(ii,iv,vi), 5(f,i,m,r,s), 6(ii,iv,vi), 7(ii,iv,vi,viii,x), 8.

\subsection*{Opgave 6.24}
% ‘Piet houdt van Annie, maar Annie houdt van een ander’.
$Hpa \wedge \E x(Hax \wedge x \neq p) $


\subsection*{Opgave 6.26}
De formule is waar. Alleen 1 heeft maar een pijl die naar 2 gaat. Van 2 gaat er
ook een pijl naar 2, maar er gaat ook een pijl vanuit 2 naar 3.


\subsection*{Opgave 6.30}
\begin{itemize}
  \item Waar      % 3
  \item Waar      % 3
  \item Niet waar % 2
  \item Waar      % 3
  \item Niet waar % 3
  \item Waar
\end{itemize}


\subsection*{Extra Opgave 5}
\begin{enumerate}[label=\alph*]
  \setcounter{enumi}{5}
  % Er zijn hoogstens drie goden.
  \item Domein: goden

  $\E x \E y \E z (x \neq y, y \neq z)$

  \setcounter{enumi}{8}
  % Een mens heeft twee ouders.
  \item Domein: mensen

  Vertaalsleutel: Oxy : x Ouder van y

  $\A k \E m \E p (Omk, Opk)$

  \setcounter{enumi}{12}
  % Karel houdt alleen van hen van wie Elske houdt.
  \item Domein: mensen

  Vertaalsleutel: Hxy : x Houdt van y

  $\A x (Hex \rightarrow Hax)$

  \setcounter{enumi}{17}
  % Iedereen houdt alleen van zichzelf.
  \item Domein: mensen

  Vertaalsleutel: Hxy : x Houdt van y

  $\A x \A y(Hxx \wedge \neg Hxy)$

  \setcounter{enumi}{18}
  % Iemand die van iedereen houdt behalve van zichzelf is een altruist.
  \item Domein: mensen

  Vertaalsleutel: Hxy : x Houdt van y, Ax : x is een Altruist

  $\A x \A y (Hxy \wedge Hxx)$

\end{enumerate}


\subsection*{Extra Opgave 6}
\begin{enumerate}[label=\roman*]
  \setcounter{enumi}{1}
  % Er zijn geen symmetrische relaties
  \item Niet waar

  \setcounter{enumi}{3}
  % Er zijn twee nodes, er bestaat geen parent voor een van de twee.
  \item Waar

  \setcounter{enumi}{5}
  % Er zijn twee nodes, waarvanuit een pijl gaat naar een z. Deze twee nodes,
  % zie naar de z gaan, als ze naar een z gaan, gaat de ander ook naar die z.
  \item Niet waars

\end{enumerate}


\subsection*{Extra Opgave 7}
\begin{enumerate}[label=\roman*]
  \setcounter{enumi}{1}
  % Als x en y twee lijnen zijn, dan bestaat er een punt die op die lijnen ligt.
  \item Waar  % Punt 2

  \setcounter{enumi}{3}
  % Voor alle punten zijn er twee lijnen waarbij ieder punt of een van de twee
  % lijnen ligt.
  \item Waar

  \setcounter{enumi}{5}
  % Er bestaan twee lijnen en twee punten waarbij beide punten op beide lijnen
  % liggen.
  \item Niet waar

  \setcounter{enumi}{7}
  % Alle lijnen hebben drie punten die op de lijnen liggen, en de punten liggen
  % niet op elkaar.
  \item Niet waar

  \setcounter{enumi}{9}
  % Alle punten die op twee lijnen liggen, liggen tussen twee punten in.
  \item Niet waar

\end{enumerate}

\end{document}
