
% Read ”5. Part 2 What does fuzzy logic have to offer?” on pages 2769-2770 of the paper by Lotfi Zadeh (Paper 1) given in Section 2. Choose one contribution (e.g. Computational theory of perceptions) and write a comprehensive and concise essay (300-500 words) that is guided by the points outlined in Sect. 1.


% (a) What is the main point of the text in your own words? How do you think it is developed?
Machines cannot think like humans do, yet. In order to make a machine perform tasks, it needs clear instructions. As Zadeh states: "In a one-way communication via natrual language between a human (sender) and a machine (recipient), mm-precisiation is a necessity becausee a machine cannot understand unprecisiated natural language." \cite[2760]{is_there_a_need_for_fuzzy_logic}. Because reality is fuzzy, and most concepts in science are fuzzy, tasks need to be translated for machines to understand.
Zadeh explains the different modalities of precisiation and expands on the fact that bivalent logic is often not cointensive. These concepts are explained regorously in chapter three of the paper.
% (b) Identify the purpose and impact of the text and express them in your own words.
The purpose of the text is to convince the reader of the value that fuzzy logic may offer over a bivalent-logic-based approach, and that the implications are important for science. \\
% (c) What idea(s) stood out to you? Why? Were they completely new to you or were they in opposition to existing scholarship that you are aware of or you have used before? (You may either focus on one idea or refer to the entire selected section and discuss 2-3 ideas that enhance your understanding.)
The definitions used in this text were new to me.


In the text "Computational theory of perceptions" \cite[2770]{is_there_a_need_for_fuzzy_logic}, Zadeh




% (d) What are the observations, opinions or experiences that shape your understanding of this text?





% (e) Do you agree or disagree with this text/argument? Why? Give elaborate reasoning.
Zadeh states that many concepts in science are a matter of degree, and that therefore bivalent-logic-based definitions of scientific concepts are not cointensive \cite[2769]{is_there_a_need_for_fuzzy_logic}, meaning that the precisiated meaning is not close from the actual meaning. Then he states that fuzzy logic is a necessity to formulate cointensive definitions of fuzzy concepts.



% (f) How does this text reinforce your existing ideas? How does this text challenge your existing ideas or assumptions?

The text uses explicit terms, that were often unfamiliar to me, to describe important concepts, forcing me to investigate and broaden my knowledge. The word `precisiation' may not be found in a dictionary, but Zadeh has defined, used, and expanded upon the concept in preceding chapters and other writings such as \cite{concept_of_cointensive_precisiation}. Another term `cointension' is explained in detail \cite[2760]{is_there_a_need_for_fuzzy_logic}.


% (g) How does this text help you to better understand this topic or explore this field of study/discipline?






