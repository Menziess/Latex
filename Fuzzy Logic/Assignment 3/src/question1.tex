
% Read ”5. Part 2 What does fuzzy logic have to offer?” on pages 2769-2770 of the paper by Lotfi Zadeh (Paper 1) given in Section 2. Choose one contribution (e.g. Computational theory of perceptions) and write a comprehensive and concise essay (300-500 words) that is guided by the points outlined in Sect. 1.


% (a) What is the main point of the text in your own words? How do you think it is developed?
Machines cannot think like humans do. As Zadeh states: "In a one-way communication via natural language between a human (sender) and a machine (recipient), mm-precisiation is a necessity because a machine cannot understand unprecisiated natural language." \cite[2760]{is_there_a_need_for_fuzzy_logic}. Because reality is fuzzy, and most concepts in science are fuzzy, a concept need to be translated before machines will be able to understand its meaning.
Zadeh explains the different modalities of precisiation and expands on the fact that bivalent logic is often not cointensive. These concepts are explained rigorously in chapter three of the paper.
% (b) Identify the purpose and impact of the text and express them in your own words.
The purpose of the text is to convince the reader of the value that fuzzy logic may offer over a bivalent-logic-based approach, and the importance of the implications for science. \\

% (c) What idea(s) stood out to you? Why? Were they completely new to you or were they in opposition to existing scholarship that you are aware of or you have used before? (You may either focus on one idea or refer to the entire selected section and discuss 2-3 ideas that enhance your understanding.)
The idea that science is bivalent-logic-based stood out to me. It's true that an hypothesis is proven to be true or false, but the formulation of the hypothesis itself may contain degrees or uncertainties. I'm left wondering why it would be useful to determine the degree to which a proven statement is true or false in science. \\
% (d) What are the observations, opinions or experiences that shape your understanding of this text?

In the fifth lecture of Fuzzy Logic, we discussed uncertainty, human reasoning and linguistic modelling. During this lecture, we had a discussion about probability that closely resembles my doubts on this matter. A drink had a membership degree equal to the other drinks probability of being potable. The question was: "You must drink from the one you choose. Which would you choose to drink from?" \cite[slide. 9]{lecture_5_uncertainty}. If a drink is potable with a degree of 0.91, is it safe to drink? What about a degree of 0.500001, or 0.499999? Uncertainty could be stated in the hypothesis, for example: given a bottle of water with 0.01\% poison, would it be potable? The proposition would by bivalence be true or false. In this case the answer would be more valuable than a degree of truthfulness. \\
% (e) Do you agree or disagree with this text/argument? Why? Give elaborate reasoning.
Zadeh states that in large measure, science is bivalent-logic-based, and that because most concepts in science are fuzzy, the bivalent-logic-based definitions are not cointensive \cite[2769]{is_there_a_need_for_fuzzy_logic}. I agree that this is the case, but I doubt that fuzzy logic is a necessity when it comes to formalizing cointensive definitions of concepts in science. He seems to label scientific concepts with a matter of degree as fuzzy concepts. I also have a strong doubt that scientific questions should be answered with an associated degree of certainty. \\

% (f) How does this text reinforce your existing ideas? How does this text challenge your existing ideas or assumptions?
The text uses explicit terms, that were often unfamiliar to me, to describe important concepts, forcing me to investigate and broaden my knowledge. The word `precisiation' may not be found in a dictionary, but Zadeh has defined, used, and expanded upon the concept in preceding chapters and other writings such as \cite{concept_of_cointensive_precisiation}. Another term `cointension' is rigorously explained in the second paragraph of \cite[2760]{is_there_a_need_for_fuzzy_logic}. \\

% (g) How does this text help you to better understand this topic or explore this field of study/discipline?
The text may not long enough to provide a comprehensive understanding of the topic. But it shows another contribution within the field of fuzzy logic, it's use, and it increased my intuition on the subject.
