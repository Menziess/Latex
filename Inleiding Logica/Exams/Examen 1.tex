%%%%%%%%%%%%%%%%
%% Preambule  %%
%%%%%%%%%%%%%%%%

\documentclass[11pt]{article}

\usepackage{amsmath,amsfonts,amssymb}
\usepackage{upgreek}
\usepackage{enumerate}
\usepackage{enumitem}   %
\usepackage{multicol}
\usepackage{scrextend}
\usepackage{fancyhdr}
\usepackage{lastpage}
\usepackage{subcaption} % 2 column images
\usepackage[all]{xy}
\usepackage{tikz-qtree}
\usepackage{latexsym}
\usepackage{forest}
\usepackage[margin=2.5cm]{geometry}
\input{../../Template/fitch.sty}

\setlength{\parindent}{0pt}

\pagestyle{fancy}

\lhead{\opdrachtNaam \opdrachtNummer}
\rhead{\naam(\studentNummer)}
\rfoot{Pagina\ \thepage\ van\ \pageref{LastPage}}
\lfoot{\datum}
\cfoot{}

\renewcommand\headrulewidth{0.4pt}
\renewcommand\footrulewidth{0.4pt}

\newcommand{\E}{\exists}
\newcommand{\A}{\forall}

\newcommand{\ccen}[2]{\llap{$#1$}${}\mathrel{\circ}{}$\rlap{$#2$}}

%%%%%%%%%%%%%%
%% Gegevens %%
%%%%%%%%%%%%%%

\newcommand{\naam}          {Stefan Schenk}
\newcommand{\studentNummer} {11881798}
\newcommand{\opdrachtNaam}  {Exam}
\newcommand{\opdrachtNummer}{1}
\newcommand{\datum}         {Februari 2018}

%%%%%%%%%%%%%%%%
%% Antwoorden %%
%%%%%%%%%%%%%%%%

\begin{document}

% Examen 1:

\subsection*{Opgave 1}
Gegeven zijn de verzameling S van AI-studenten en L van laptops. Tussen deze verzamelingen gelden de relaties $B \subseteq S \times L$ (bezitten) en $G \subseteq S \times L$ (gebruiken). Formaliseer onderstaande frases in de taal van de verzamelingenleer. (Tip: je mag domein en bereik (range) gebruiken.)

\begin{enumerate}[label=\alph*]

  \item Iedere AI-student gebruikt de laptops die hij of zij bezit.

  \item Iedere AI-student bezit een laptop.

  \item Er zijn AI-studenten die laptops gebruiken die van niemand zijn.

\end{enumerate}


\subsection*{Opgave 2}
Gegeven zijn de volgende verzamelingen:

\[ A=\{a, \emptyset\} \quad B=\{b, \{a,\emptyset\}\} \quad C=\{a, b\}\]

\begin{enumerate}[label=\alph*]

  \item Welke van de volgende uitspraken zijn waar en welke niet?

        \[ a \in A, \quad a \in B, \quad A \in B, \quad A \subseteq B\]

  \item Geef de volgende verzamelingen door hun elementen op te sommen.

		\[ B \cap C, \quad A \times C, \quad C-B, \quad \mathcal{P}(A)\]

\end{enumerate}


\subsection*{Opgave 3}
Beschouw de volgende relatie R over de verzameling $A = \{0, 1, 2, 3\}$.

	\[ R=\{ \langle0, 3\rangle,
			\langle1, 0\rangle,
			\langle3, 2\rangle\}\]

\begin{enumerate}[label=\alph*]

	\item Ga na of de relatie $R$ reflexief, irreflexief, symmetrisch, asymmetrisch en/of antisymmetrisch is. Beargumenteer je antwoord!

	\item De relatie $R$ is niet transitief. Welke paren moeten we voegen aan $R$ om een transitieve relatie te krijgen? (Noem alleen paren die absoluut noodzakelijk zijn om van $R$ een transitieve relatie te maken.)

\end{enumerate}


\subsection*{Opgave 4}
Zij $\{0,1\}*$ de verzameling van eindige rijtjes bestaande uit de symbolen 0 en 1. (Typische elementen zijn dus 011000, 100010 en 1010100.) We definieren inductief een verzameling $A \subseteq \{0, 1\}*$, als volgt:

\begin{description}[font=$\bullet$\scshape\bfseries]

	\item $011 \in A$.

	\item Als $\sigma \in A$, dan ook $\sigma\sigma \in A$ en $11\sigma0 \in A$. (Dat wil zeggen, als $\sigma$ in A zit, dan zit het rijtje dat je krijgt door twee keer $\sigma$ achter elkaar op te schrijven ook in A; en als $\sigma$ in A zit, zit het rijtje dat je krijgt door er 11 voor te zetten en er een 0 achter te zetten ook in A.)

	\item Niets anders zit in A.
\end{description}

\begin{enumerate}[label=\alph*]

	\item Definieer met recursie functies $t_0 : A \rightarrow N$ en $t_1 : A \rightarrow N$ waarbij $t_i (\sigma)$ het aantal keer is dat $i \in \{0, 1\}$ voorkomt in $\sigma \in A$.

	\item Bewijs met inductie dat $t_1 (\sigma) = 2 \cdot t_0 (\sigma)$ voor alle $\sigma \in A$.
\end{enumerate}


\subsection*{Opgave 5}
Vertaal de volgende zinnen in de taal van de propositielogica. Geef zoveel mogelijk de logische structuur weer en vermeld de vertaalsleutel.

\begin{enumerate}[label=\alph*]

	\item Als het alarm afgaat, dan is er echt brand of er is weer een oefening.

	\item Hoewel het team sterker is dan vorig jaar, zal het toch ook dit jaar geen kampioen worden.

	\item Baat het niet dan schaadt het niet.

\end{enumerate}


\subsection*{Opgave 6}
Onderzoek in een waarheidstafel de volgende redeneerschema's op hun geldigheid. In geval van ongeldigheid wijs een specifieke interpretatie aan die laat zien dat het schema niet geldig is.

\begin{enumerate}[label=\alph*]

	\item $(p\wedge\neg q)\rightarrow r \models (\neg r) \rightarrow (p\vee q)$

	\item $p\rightarrow(q\vee r), (\neg q) \leftrightarrow r \models \neg p$

\end{enumerate}


\subsection*{Opgave 7}
Bewijs met behulp van natuurlijke deductie dat de volgende redeneringen geldig zijn.
\begin{enumerate}[label=\alph*]

	\item $(p\vee q) \rightarrow r \vdash (p \rightarrow r) \wedge (p \rightarrow r)$

	\item $r \rightarrow \neg(p \vee q), (\neg p) \rightarrow q \vdash \neg r$

\end{enumerate}

\subsection*{Opgave 8}
\begin{enumerate}[label=\alph*]

	\item Bewijs of weerleg de geldigheid van de redenering $\neg(p\rightarrow q), r\rightarrow(p\vee q) \models \neg r$ met behulp van Beth tableaus. Als de redenering geldig is, laat zien dat het tableau sluit. Als de redenering ongeldig is, geef dan een tegenvoorbeeld en laat zien hoe je dit tegenvoorbeeld uit je tableau kunt aflezen.

	\item Laat met behulp van Beth tableaus zien dat de formule

		\[\neg(p\wedge\neg q) \rightarrow (p \rightarrow q)\]

		een tautologie is.

\end{enumerate}

\end{document}
